\documentclass[12pt]{article}
\usepackage{a4wide}
\usepackage{NotationStyle}
\usepackage[]{algorithm2e}
%\usepackage[noend]{algorithmic}
\usepackage[english]{babel}
\usepackage{amsmath,amssymb,mathrsfs,mathtext}
\usepackage{graphics,graphicx,epsfig}
\usepackage{epstopdf}
\usepackage{fancybox,fancyhdr}
\usepackage{enumerate}
\usepackage{array}
\usepackage{color, soul}
\usepackage[normalem]{ulem}
\usepackage{arydshln}
\setlength\dashlinedash{0.2pt}
\setlength\dashlinegap{4.5pt}
\setlength\arrayrulewidth{0.2pt}

\newcommand{\argmax}{\mathop{\rm arg\,max}\limits}
\newcommand{\dtr}{\Delta t_\text{r}}
\newcommand{\dtp}{\Delta t_\text{p}}

\begin{document}
\section{Strategies of martix construction}
Suppose a large set of time series~$\fD=\{\bs\}$ is given. The ``object-feature'' matrix~$\bX^*$ for the multiscale autoregressive problem statement is composed of row-vectors
\[
\bs_i^\prime = [\by_i^\prime, \bx_i^\prime] = [\underbrace{s(t_i),\dots,s(t_i-\Delta t_\text{r})),}_{\by_i^\prime}
\underbrace{\dots,s(t_i-\Delta t_\text{r}-\Delta t_\text{p})}_{\bx_i^\prime}],
\]
where~$s(t)$ is an element of time series~$\bs$. Consider several strategies to decompose time series $\bs$ into segments $\Delta t_i = (t_i,\dots,t_i-\Delta t_\text{r}-\Delta t_\text{p})$ to construct  matrix $\bX^{*}$.
\begin{enumerate}
\item Row vectors $\bs_i$ cover time series without intersections. Let $\{T_{\max}, \dots, 1\}$ be the set of indices of tim series $\bs$, then thee strategy of selecting $t_i$ holds the following:
\begin{equation}\label{eq:strategy1}\{T_{\max}, \dots, 1\} = \bigsqcup_{i=1}^{M} \Delta t_i.\end{equation}
It follows from~\eqref{eq:strategy1} that $|t_{i+1} - t_i| > \dtr + \dtp$ for any $i = 1, \dots, M-1$.
\item Row vectors $\bs_i = [\y_i, \x_i]$ overlap, but target parts  $\y_i$ do not intersect:
    \begin{equation}\label{eq:strategy2}\{T_{\max}, \dots, 1\} = \bigsqcup_{i=1}{M-1} \{t_i,\dots,t_i-\Delta t_\text{r}\} \Rightarrow |t_{i+1} - t_i| > \dtr. \end{equation}
\item For each time stamp $t_i$ of the least frequent regular sampling there is correspondent row vector $\bs_i$ in $\bX^{*}$:
    \[\{T_{\max}, \dots, 1\} = \bigcup_i t_i. \]
\item Time intervals $\Delta t_i$ are selected randomly.
\item Potentially, other sensible strategies are possible.    
\end{enumerate}
 Vector $\veps \in \mathbb{R}^{\dtr}$ of model residuals at time stamp $t_i$ is given by
\[\veps_i = {\y}_i - \hat{\y}_i.\]
Dependent on the way the matrix $\mathbf{X}^{*}$ is designed, there might be dependencies between components of subsequent vectors $\y_{i}$, $\y_{i+1}$.
If there are such $i, \; i' \in \cB$ that $|t_i - t_{i'}| < \dtr$, vectors $\veps_i$ and $\veps_{i'}$ overlap or contain residuals for the same time stamp. In this case define the test vector of residuals as
\[\veps(\cB) = \left\{\bar{\varepsilon}_t \left| t \in \bigcup_{i \in \cB} \{i-\dtr, \dots, i\} = \{t_{i_{\min}} - \dtr, \dots, t_{i_{\max}}\}\right.\right\}, \]
where $\bar{\varepsilon}_t$ is the average residual for the time stamp $t$.

To avoid these issues, we fix the second strategy of the $\bX$ construction.
\section{Forecast analysis}
We consider the following testing procedure, given by the algorithm~\ref{alg:train_test_rmse}:

\begin{algorithm}[!h]
\textbf{\emph{ComputeForecastingErrors()}}

 \KwData{$\bX^{*} \in \mathbb{R}^{M\times(\dtr+\dtp)}$. Parameters: number of testing procedures $N$, train to test ratio $\alpha$.}
 \KwResult{Forecasting quality: root-mean-squared error.}
 $n = 1$\;
 \While{$n < N$:}{
 define  $m = \lfloor M/N\rfloor$, $\bX^{*}_n = [\bx^{*}_{(n-1)m +1}, \dots, \bx^{*}_{nm}]\T$ \;
  $\bX_{\text{train}}, \; \bX_{\text{test}}, \;\bX_{\text{val}} = TrainTestSplit(\bX^{*}_n, \alpha)$\; 
  train forecasting model $\fx(\x, \hat{\w}_n)$, using $\bX_{\text{train}}$ and $\bX_{\text{test}}$\;
  obtain vector of residuals $\veps = [\varepsilon_{T}, \dots, \varepsilon_{T- \dtr + 1}]$ with respect to $\bX_{\text{val}}$ \;
  compute forecasting quality:
  \[ \text{RMSE}(n)  = \sqrt{\frac{1}{\dtr} \sum_{t=0}^{\dtr} \varepsilon_{T-t}^2};\]
  $n = n + 1$ \;
  }
  Average RMSE by data splits.
  \bigskip
 
\textbf{\emph{TrainTestSplit()}}
 
\KwData{Object-feature matrix $\bX^{*} \in \mathbb{R}^{m\times(\dtr+\dtp)}$. Train to test ratio $\alpha$.}
 \KwResult{Train, test, validation matrices $\bX^{*}_{\text{train}}$, $\bX^{*}_{\text{test}}$, $\bX^{*}_{\text{val}}$.}
 Set train set and test set sizes:
 
 $ \quad m_{\text{train}} = \lfloor\alpha(m-1)\rfloor$ \;
 $ \quad m_{\text{test}} = m - 1 - m_{\text{train}} $ \; %(1-\alpha)(m-1)
 Decompose matrix $\bX^{*}$ into train, test, validation matrices $\bX^{*}_{\text{train}}$, $\bX^{*}_{\text{test}}$, $\bX^{*}_{\text{val}}$:
 \[\bX^{*}_{\text{train}} = \left[\begin{array}{c} \x^{*}_{\text{val}} \in \mathbb{R}^{1\times (\dtr + \dtp)}\\
 \hline
 \bX^{*}_{m_{\text{test}}} \in \mathbb{R}^{m_{\text{test}} \times (\dtr + \dtp)} \\
 \hdashline
 \bX^{*}_{m_{\text{train}}} \in \mathbb{R}^{m_{\text{train}} \times (\dtr + \dtp)} 
 \end{array}\right] = \left[\begin{array}{c|c} \y_{\text{val}} & \x_{\text{val}} \\
 \hline 
 \bY_{m_{\text{test}}} & \bX_{m_{\text{test}}}  \\
 \hdashline
 \bY_{m_{\text{train}}}  & \bX_{m_{\text{train}}}
 \end{array}\right]
 \]
 %$\bX^{*}_{\text{train}} = \left[\begin{array}{c|c} \y_1 & \x_1 \\ \dots &\dots, \\
% \y_{m_{\text{train}}} & \x_{m_{\text{train}}} \end{array}\right], ~~
% \bX^{*}_{\text{test}} = \left[\begin{array}{c|c} \y_1 & \x_1 \\ \dots & \dots, \\
% \y_{m_{\text{test}}}& \x_{m_{\text{test}}} \end{array}\right] $ \;
 \caption{Train-test split.}\label{alg:train_test_rmse}
\end{algorithm}

\subsection{Ensuring forecast model validity}
A valid forecast model must meet the following conditions:
\begin{itemize}
\item Mean of residuals equals to zero.
\item Residuals are stationary.
\item Residuals are not autocorrelated.
\end{itemize}
If the forecast does not meet any of these conditions, then it can be further improved by
 simply adding a constant (minus residual mean) to the model, balancing variance or including more lags. Additionally, desirable properties are normality and homoscedasticity of residuals. These properties are not necessary for an adequate model, but allow to obtain theoretical estimations of the confidence interval.

\subsection{Forecasting quality}

\end{document} 

\paragraph{Step-by-step forecasts.}
Suppose that a forecasting model $\fx$ was built, producing forecasts
\[\hat{\y}_i = [\hat{s}(t_{i}), \dots, \hat{s}(t_{i}- \dtr)] = \fx(\x_i, \w).\]
In traditional step-by-step forecasting scheme $(k+1)$-th component of $\y_i$ depends on the forecasts of the previous $k$ components.
\[\hat{s}(t_{i}-k)  = f(\hat{\x}_i^{(k)}, \w), \quad k = 0,\dots, \dtr-1,\]
where $\hat{\x}_i^{(k)}$ includes forecasts of $k$  components of $\y_i$:
\[\hat{\x}_i^{(0)} = \x_i = [s(t_i - \dtr - 1), \dots, s(t_i - \dtr - \dtp)], \]
\[\hat{\x}^{(1)} = [\hat{s}(t_i - \dtr), s(t_i - \dtr - 1), \dots, s(t_i - \dtr - \dtp + 1)], \]
\[\dots\]
\[\hat{\x}_i^{(k)} = [\underbrace{\hat{s}(t_i - \dtr + k-1), \dots, \hat{s}(t_i - \dtr)}_{k}, s(t_i - \dtr - 1), \dots, s(t_i - \dtr - \dtp + 1)], \]
